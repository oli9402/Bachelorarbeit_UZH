\begin{abstract}
\setstretch{1.25}

% CONTENT OF ABS HERE--------------------------------------
The number of people living with dementia is estimated to triple by the year 2050. This makes Alzheimer's Disease (AD), which accounts for up to 80\% of all dementia cases, a crucial area of study. With the trend of biomarkers gaining importance in diagnosing AD, imaging techniques such as magnetic resonance imaging (MRI) have been widely studied. To find disease related patterns, machine learning algorithms have been applied. Furthermore, the interest in automated classification systems assisting AD diagnosis has increased. Support vector machine (SVM) and convolutional neural network (CNN) are the most often used machine learning algorithms for automated classification. Therefore, this review included 10 studies using either SVM or CNN for automatic AD classification with the goal of comparing them. 
It can be concluded that although both algorithms perform well when classifying AD vs. normal cognition, direct comparison between the two is limited since size of data set, pre-processing steps and evaluation methods vary. Novel research provided a framework to make comparison more transparent. Moreover, all reviewed studies relied on the ADNI database to train their models, which may influence generalizability.      

% END CONTENT ABS------------------------------------------
\noindent
%\textit{\textbf{Keywords: }%
%key1; key2; key3; key4.} \\ %% <-- Keywords HERE!


\end{abstract}
