\section{Introduction} \label{introduction} 


\\


%%%Prävalenz 
The Alzheimer Europe Organisation estimates that by the 2050 the amount of people living with dementia will have doubled in Europe, meaning an increase from 9'780'678 to 18'846'286 people \autocite{Europe}. Other estimations suggest that dementia is present in 50 million people worldwide and is predicted to triple by the year 2050 \autocite{AZ}. \gls{AD} is the most prevalent form of dementia and may account for 60-80\% of all dementia cases \autocite{8267050}. In elder population, \gls{AD} has a prevalence of 10\% and is considered the most common neurodegenerative disease \autocite{nanniComparisonTransferLearning2020}.

%-- labeled after the suggested biological markers for \gls{AD}: $\beta$-amyloid, tau and neurodegeneration \autocite{jackNIAAAResearchFramework2018} --

Diagnostic criteria for \gls{AD} have changed multiple times, moving past an exclusively clinical approach of diagnosing \gls{AD}, towards incorporating biological aspects \autocite{SCHELTENS2021}. The \gls{nia-aa} putting forward the AT(N) research framework -- labeled after the suggested biological markers for \gls{AD}: $\beta$-amyloid, tau and neurodegeneration \autocite{jackNIAAAResearchFramework2018} -- reflects the trend of separating clinical symptoms from disease pathology and having purely biological criteria for \gls{AD} \autocite{SCHELTENS2021, ahmedHistoryPerspectiveHow2021}. Still, diagnosis is primarily done clinically \autocite{ahmedHistoryPerspectiveHow2021} and, as \textcite{islamBrainMRIAnalysis2018} describe, requires physical and neurological examinations, a detailed history of the patient as well as tests, like \gls{mmse} \autocite{FOLSTEIN1975189}. A definite diagnosis can only be made post-mortem \autocite{nanniComparisonTransferLearning2020}.

In vivo diagnosis accuracy can be affected by the heterogeneity of the clinical symptoms in \gls{AD}  \autocite{ahmedHistoryPerspectiveHow2021, frisoni2011} but increases with the development of the disease \autocite{islamBrainMRIAnalysis2018}. Still, it is notable that \enquote{\textit{10\%} to \textit{30\%} of individuals clinically diagnosed as \gls{AD} dementia by experts do not display \gls{AD} neuropathologic changes at autopsy} \autocite[Nelson, 2011, as cited in][p. 538]{jackNIAAAResearchFramework2018}. Although, as \textcite{jackNIAAAResearchFramework2018} state, the AT(N) framework isn't ready for general clinical usage, it addresses these mentioned challenges by defining \gls{AD} biologically. This definition includes biomarkers from three categories that form the acronym AT(N): $\beta$-amyloid accumulation, pathologic tau and neurodegeneration \autocite{jackNIAAAResearchFramework2018}. Compared to \gls{pet}, that measures $\beta$-amyloid accumulation or tau, \gls{mri} is less invasive and can be used to measure neurodegeneration \autocite{wangDenseCNNDenselyConnected2021}. Additionally, \gls{mri} is always recommended following the clinical evaluation \autocite{SCHELTENS2021} and is often used as a tool for finding relevant biomarkers \autocite{liuMultimodelDeepConvolutional2020a}. 

Methods such as mass-univariate analyses which are large amount of univariate tests (e.g. t-tests) \autocite{groppeMassUnivariateAnalysis2011} have been used in combination with neuroimaging studies and led to significant insight into psychiatric and neurological disorders \autocite{vieiraUsingDeepLearning2017}. In the case of \gls{mri} analysis of \gls{AD}, hippocampus and other medial temporal lobe structures are of interest \autocite{islamBrainMRIAnalysis2018,SCHELTENS2021}. 
\textcite{vieiraUsingDeepLearning2017} describe \gls{ML} methods as alternative analysis techniques which, in comparison, are multivariate meaning that they take the inter-correlation between voxels into consideration \autocite{vieiraUsingDeepLearning2017}. Furthermore, \gls{ML} methods could allow for individual  treatment decisions to be made \autocite{naikDenouementsMachineLearning2020}, since these methods make statistical inference at an individual level \autocite{vieiraUsingDeepLearning2017}. These properties have created a growing interest in \gls{ML} methods for analysing neuroimaging data \autocite{vieiraUsingDeepLearning2017}. Furthermore, \gls{ML} methods could automate the process of analysing \gls{mri} images and consequently safe resources \autocite{islamBrainMRIAnalysis2018}. Additionally, \gls{ML} methods have the ability of finding disease-related patterns without prior knowledge about underlying mechanisms \autocite{nanniComparisonTransferLearning2020}. 

Considering these described aspects, \gls{ML} methods seem suitable for analysing MRI scans related to \gls{AD}. Many studies using \gls{ML} to classify \gls{AD} have been reviewed in the past \autocite[see][]{ebrahimighahnaviehDeepLearningDetect2020,joDeepLearningAlzheimer2019, tanveerMachineLearningTechniques2020, wenConvolutionalNeuralNetworks2020, noorApplicationDeepLearning2020}. Aiming to provide an updated view on the state of research, this review compares novel studies using \gls{ML} methods for classifying Alzheimer's Disease.







