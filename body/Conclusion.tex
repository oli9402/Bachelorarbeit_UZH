\section{Conclusion} \label{Discussion} 
This review showed that both the \gls{svm} and \gls{cnn} achieve good performance in the binary classification task (\gls{AD} vs. \gls{nc}). A known problem with training data (e.g. relying on one data base) is a limitation that was found in this review. Furthermore, over fitting  seems to be a problem that was more present in the studies using \gls{svm}. To conclude what model is better is a difficult question that can't be answered with this review and further research needs to be done in this regard (see table \ref{tab:pro} for their strengths and weaknesses). A big constraint holding back comparison is the fact that evaluation needs to be more transparent. Frameworks \autocite[e.g.][]{wenConvolutionalNeuralNetworks2020} and challenges \autocite[e.g.][]{bron} need to considered in future research. 
A limitation of this review is the small number of included studies and especially studies using \gls{cnn} are unrepresentative. On the other hand, this reflects the fact that the focus of research in the last two years research was on \gls{DL} methods such as \gls{cnn}.
Lastly, the diagnosis of \gls{AD} has its limitations and since both methods reviewed in this paper are supervised learning methods, this may need to be addressed in future research. 